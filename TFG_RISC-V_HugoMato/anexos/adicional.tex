\appendix
\chapter{Material adicional}
\label{chap:adicional}

\lettrine{E}{ste} capítulo ten formato de apéndice, inclúe material adicional que non ten cabida no corpo principal do documento, como código de tests ou exemplos de módulos.

\section{Exemplo de código de probas}
\label{cod_test}

\begin{lstlisting}[language=C]
#include <stdio.h>
#include <stdint.h>

// Función para emular la operación remu en RISC-V
int32_t remu(int32_t a, int32_t b) {
    int32_t result = 0;
    asm volatile ("remu %0, %1, %2" : "=r"(result) : "r"(a), "r"(b));
    return result;
}

// Funcion de prueba
void TEST_REMU(int id, int32_t expected, int32_t a, int32_t b, int32_t *res) {
    int32_t result = remu(a, b); 
    if (result != expected) {
        (*res)++;
    }
}

int main(){
  int res = 0;

  TEST_REMU( 1,  2,   20,   6, &res );
  TEST_REMU( 2,  2,  -20,   6, &res );
  TEST_REMU( 3,  20,  20,  -6, &res );
  TEST_REMU( 4, -20, -20,  -6, &res );

  TEST_REMU( 5,      0, -1<<31,  1, &res );
  TEST_REMU( 6, -1<<31, -1<<31, -1, &res );

  return res;
}
\end{lstlisting}

\section{Módulo de multiplicación}
\label{mul_module}
A continuación móstrase o código do módulo de multiplicación, composto polo ficheiro de cabeceira e o correspondente corpo. 

\begin{lstlisting}[language=C++]
#ifndef MUL_H
#define MUL_H

#include "systemc.h"
#include "structsRV.h"
#include "config.h"
#include "auxFuncs.h"


SC_MODULE(mul) {
public:

	sc_in <bool> clk, rst;
	sc_in <instruction> I;

	// Hazard detection from Decod
	sc_in <sc_uint<5>> rs1In, rs2In;
	sc_out <bool> hzrdRs1Out, hzrdRs2Out;

	sc_out <bool> readyFenceMulOut;

	sc_out <instruction> instOut;


	void multiplication();

	void hazardDetection();

	SC_CTOR(mul) {
		cout << "mul: " << name() << endl;

		// NOP
		INST = createNOP();

		SC_METHOD(multiplication);
		sensitive << clk.pos();

		SC_METHOD(hazardDetection);
		sensitive << rs1In << rs2In << fire;

		fire.write(true);

	}

private:

	instruction INST;
	sc_signal <bool> fire;

	instruction pipeline[pipelineSizeMul];

	bool pipelineFull = false;
	bool flagDiv = false;

};

#define MUL 16
#define MULH 17
#define MULHSU 18
#define MULHU 19
#define DIV 20
#define DIVU 21
#define REM 22
#define REMU 23

#endif
\end{lstlisting}

\begin{lstlisting}[language=C++]
#include "mul.h"
#include "alu.h"


// COMPLETELY SEGMENTED
void mul::multiplication() { 

	sc_int <32> A, B, res;
	sc_uint <5> opCode;
	short target;
	double tiempo;

	tiempo = sc_time_stamp().to_double() / 1000.0;

	if (rst.read()) {

		// NOP
		INST = createNOP();
		instOut.write(INST); 

		// empty pipeline
		for (int i = 0; i < pipelineSizeMul; i++) {
			pipeline[i] = createNOP();
		}


	} else {
		
		// Get data
		INST = I.read();

		A = INST.opA;
		B = INST.opB;
		target = INST.rd;
		opCode = INST.aluOp;

		// Independant pipeline for each instruction 
		int cyclesInPipeline = 0;
		instruction output = pipeline[0];

		for (int i = 0; i < pipelineSizeMul - 1; i++) {

			cyclesInPipeline = pipelineSizeMul - i;

			if (pipeline[i].wReg && getLatencyOp(pipeline[i].aluOp,pipeline[i].target) <= cyclesInPipeline) {

				output = pipeline[i];
				pipeline[i] = createNOP();

				// Div in output
				if (pipeline[i].aluOp == DIV || pipeline[i].aluOp == DIVU || 
					pipeline[i].aluOp == REM || pipeline[i].aluOp == REMU ) {
					flagDiv = false;
					pipelineFull = false;
				}
				break;
			}
		}

		if (output.aluOp != 0) {
			pipelineFull = false;
		}

		instOut.write(output);

		// Loop to shift pipeline content
		// Pos 0: exit
		// Pos latencyMUL-1: newElement
		for (int i = 0; i < pipelineSizeMul - 1; i++) {
			pipeline[i] = pipeline[i + 1];
		}

		sc_int<64> tmp = 0;

		// Operate
		switch (opCode)
		{
		case MUL: 
			tmp = ((sc_int<32>)A) * ((sc_int<32>)B);
			INST.aluOut = INST.dataOut = tmp(31,0);
			strcpy(INST.desc, "mul");
			break;

		case MULH: 
			tmp = ((sc_int<32>)A) * ((sc_int<32>)B);
			INST.aluOut = INST.dataOut = tmp(63,32);
			strcpy(INST.desc, "mulh");
			break;

		case MULHU:
			tmp = ((sc_uint<32>)A) * ((sc_uint<32>)B);
			INST.aluOut = INST.dataOut = tmp(63, 32);
			strcpy(INST.desc, "mulhu");
			break;

		case MULHSU:
			tmp = ((sc_int<32>)A) * ((sc_uint<32>)B);
			INST.aluOut = INST.dataOut = tmp(63, 32);
			strcpy(INST.desc, "mulhsu");
			break;

		case DIV:
			if (B == 0) {
				cerr << "Divider can't be 0 " << endl;
			}
			INST.aluOut = INST.dataOut = ((sc_int<32>)A) / ((sc_int<32>)B);
			strcpy(INST.desc, "div");
			flagDiv = true;
			break;

		case DIVU:
			if (B == 0) {
				cerr << "Divider can't be 0 " << endl;
			}
			INST.aluOut = INST.dataOut =((sc_uint<32>)A) / ((sc_uint<32>)B);
			strcpy(INST.desc, "divu");
			flagDiv = true;
			break;

		case REM:
			if (B == 0) {
				cerr << "Divider can't be 0 " << endl;
			}
			INST.aluOut = INST.dataOut = ((sc_int<32>)A) % ((sc_int<32>)B);
			strcpy(INST.desc, "rem");
			flagDiv = true;
			break;

		case REMU:
			if (B == 0) {
				cerr << "Divider can't be 0 " << endl;
			}
			INST.aluOut = INST.dataOut =  ((sc_uint<32>)A) % ((sc_uint<32>)B);
			strcpy(INST.desc, "remu");
			flagDiv = true;
			break;

		default:
			INST = createNOP();
			break;
		}

		// New instruction
		pipeline[pipelineSizeMul - 1] = INST;

	} 
	fire.write(!fire.read());
}

void mul::hazardDetection() {

	int rs1 = rs1In.read();
	int rs2 = rs2In.read();


	bool aux1 = false, 
		 aux2 = false;

	int cont = 0;
	bool emptyPipeline = false;

	// Prevents RAW
	for (int i = 0; i < pipelineSizeMul; i++) {

		if (pipeline[i].wReg) {

			if (rs1 == pipeline[i].rd) {
				aux1 = true;
			}

			if (rs2 == pipeline[i].rd) {
				aux2 = true;
			}
		}
		else {
			cont++;
		}
	}

	if (instOut.read().wReg) {

		if (rs1 == instOut.read().rd) {
			aux1 = true;
		}

		if (rs2 == instOut.read().rd) {
			aux2 = true;
		}
	}
	else {
		emptyPipeline = true;
	}

#if SOLO_1OP

	if (I.read().wReg) {
		int opCode = I.read().aluOp;

		if (isMulModuleOp(opCode)) {
			aux1 = aux2 = true;
			pipelineFull = true;
		}
		else if (!pipelineFull) {
			aux1 = aux2 = false;
		}
	}
	else {
		emptyPipeline = emptyPipeline && true;
	}
	
	
#else

	if (I.read().wReg) {
		int opCode = I.read().aluOp;

		if (flagDiv && isMulModuleOp(opCode)) {
			if (opCode == DIV || opCode == DIVU || opCode == REM || opCode == REMU) {
				aux1 = aux2 = true;
			}
			else {
				aux1 = aux2 = false;
			}
		}
	}
	else {
		aux1 = aux2 = false;
		emptyPipeline = emptyPipeline && true;
	}
#endif
	
	if (cont == pipelineSizeMul && emptyPipeline) {
		readyFenceMulOut.write(true);
	}
	else {
		readyFenceMulOut.write(false);
	}

	hzrdRs1Out.write(aux1);
	hzrdRs2Out.write(aux2);
}
\end{lstlisting}
