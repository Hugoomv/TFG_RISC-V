\chapter{Material adicional}
\label{chap:adicional}

\lettrine{E}{xemplo} de capítulo con formato de apéndice, onde se pode
incluír material adicional que non teña cabida no corpo principal do
documento, suxeito á limitación de 80 páxinas establecida no
regulamento de TFGs.

Exemplo de código de proba

\begin{lstlisting}[language=C]
#include <stdio.h>
#include <stdint.h>

// Función para emular la operación div en RISC-V
int32_t remu(int32_t a, int32_t b) {
    int32_t result = 0;
    asm volatile ("remu %0, %1, %2" : "=r"(result) : "r"(a), "r"(b));
    return result;
}

// Func de prueba
void TEST_REMU(int id, int32_t expected, int32_t a, int32_t b, int32_t *res) {
    int32_t result = remu(a, b); 
    if (result != expected) {
        (*res)++;
    }
}

int main(){
  int res = 0;

  TEST_REMU( 1,  2,   20,   6, &res );
  TEST_REMU( 2,  2,  -20,   6, &res );
  TEST_REMU( 3,  20,  20,  -6, &res );
  TEST_REMU( 4, -20, -20,  -6, &res );

  TEST_REMU( 5,      0, -1<<31,  1, &res );
  TEST_REMU( 6, -1<<31, -1<<31, -1, &res );

  return res;
}
\end{lstlisting}
