\chapter{Uso do simulador}
\label{chap:uso_simulador}

\lettrine{N}{este} capítulo explícase brevemente cómo empregar o simulador, explicando cómo xerar os arquivos .elf e empregar o simulador, así como interpretar o os resultados. 

O primeiro paso é empregar Segger Embedded Studio for RISC-V, aquí escribirase o código C para o programa. Despois, modificar en Project -> Compiler -> Elixir a extensión correcta, por exemplo no caso de que se realicen multiplicacións, débese cambiar de RV32I (por defecto) a RV32IM. Unha vez feito isto, débese compilar con ->. Ahora na carpeta [] están varios arquivos, entre eles o executable con extensión .elf.
