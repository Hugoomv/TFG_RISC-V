\chapter{Implementación}
\label{chap:implementacion}

\lettrine{T}{ras} haber creado o deseño, é necesario realizar a implementación. Neste capítulo, trataranse os problemas afrontados, as solucións elixidas e as ferramentas empregadas.

\section{Decisións á hora de implementar}\label{sec:decisions}
Unha vez deseñado o proxecto, o seguinte paso é a implementación. Durante este proceso, buscaranse aproximacións a problemas que non se afrontaron na etapa de deseño. Por exemplo, para a implementación da instrución Fence, da extensión Zifencei, introducíronse sinais no módulo Decod conectadas con todos os módulos. Grazas a isto, pódese saber se había algunha instrución executándose nalgún compoñente, o que permite retrasar a execución da seguinte instrución. Así, garántese que todas as instrucións acabaron, simulando a barreira.


\section{Instrucións implementadas}\label{sec:intrucions_implt}
Como se comentou no capítulo \ref{sec:obxectivos}, neste proxecto implementáronse as extensións M, F (parcialmente), Zifencei e Zicsr, ademais das funcionalidades da base RV32I. A continuación, unha lista das instrucións implementadas e a extensión á que pertencen:

\rowcolors{2}{white}{udcgray!25}
\begin{longtable}{l|r}
  \caption{Extensións e instrucións implementadas}
  \label{tab:instr_imple} \\

  \rowcolor{udcpink!25}
  \textbf{Nome da operación} & \textbf{Estado da implementación} \\\hline
  \endfirsthead

  \multicolumn{2}{c}{\tablename\ \thetable{} -- {\small \textit{(vén da páxina anterior)}}} \\
  \rowcolor{udcpink!25}
  \textbf{Nome da operación} & \textbf{Estado da implementación} \\\hline
  \endhead

  \multicolumn{2}{c}{\dotfill{\small \textit{(continúa na páxina seguinte)}}\dotfill} \\
  \endfoot

  \endlastfoot

   \multicolumn{2}{c}{\textbf{Extensión M — Multiplicación e división}} \\
    \textit{Mul} & Implementada \\
    \textit{Mulh} & Implementada \\
    \textit{Mulhsu} & Implementada \\
    \textit{Mulhu} & Implementada \\
    \textit{Div} & Implementada \\
    \textit{Divu} & Implementada \\
    \textit{Rem} & Implementada \\
    \textit{Remu} & Implementada \\

    \multicolumn{2}{c}{\textbf{Extensión F — Punto flotante en simple precisión}} \\
    \textit{Fadd.s} & Implementada \\
    \textit{Fclass.s} & Non implementada \\
    \textit{Fcvt.l.s} & Non implementada \\
    \textit{Fcvt.lu.s} & Non implementada \\
    \textit{Fcvt.s.l} & Non implementada \\
    \textit{Fcvt.s.lu} & Non implementada \\
    \textit{Fcvt.s.w} & Implementada \\
    \textit{Fcvt.s.wu} & Implementada \\
    \textit{Fcvt.w.s} & Implementada \\
    \textit{Fcvt.wu.s} & Implementada \\
    \textit{Fdiv.s} & Non implementada \\
    \textit{Feq.s} & Non implementada \\
    \textit{Fle.s} & Non implementada \\
    \textit{Flt.s} & Non implementada \\
    \textit{Flw} & Implementada \\
    \textit{Fmadd.s} & Non implementada \\
    \textit{Fmax.s} & Non implementada \\
    \textit{Fmin.s} & Non implementada \\
    \textit{Fmsub.s} & Non implementada \\
    \textit{Fmul.s} & Implementada \\
    \textit{Fmv.w.x} & Implementada \\
    \textit{Fmv.x.w} & Implementada \\
    \textit{Fnmadd.s} & Non implementada \\
    \textit{Fnmsub.s} & Non implementada \\
    \textit{Fsgnj.s} & Non implementada \\
    \textit{Fsgnjn.s} & Non implementada \\
    \textit{Fsgnjx.s} & Non implementada \\
    \textit{Fsqrt.s} & Non implementada \\
    \textit{Fsub.s} & Implementada \\
    \textit{Fsw} & Implementada \\
    \textit{C.flw} & Non implementada \\
    \textit{C.flwsp} & Non implementada \\
    \textit{C.fsw} & Non implementada \\
    \textit{C.fswsp} & Non implementada \\
    \textit{Fmv.h.x} & Non implementada \\

    \multicolumn{2}{c}{\textbf{Extensión A — Operacións atómicas}} \\
    \textit{Lr.w} & Non implementada \\
    \textit{Sc.w} & Non implementada \\
    \textit{Amoswap.w} & Non implementada \\
    \textit{Amoadd.w} & Non implementada \\
    \textit{Amoxor.w} & Non implementada \\
    \textit{Amoand.w} & Non implementada \\
    \textit{Amoor.w} & Non implementada \\
    \textit{Amomin.w} & Non implementada \\
    \textit{Amomax.w} & Non implementada \\
    \textit{Amominu.w} & Non implementada \\
    \textit{Amomaxu.w} & Non implementada \\

    \multicolumn{2}{c}{\textbf{Extensión Zicsr — Acceso a rexistros CSR}} \\
    \textit{Csrrw} & Implementada \\
    \textit{Csrrs} & Implementada \\
    \textit{Csrrc} & Implementada \\
    \textit{Csrrwi} & Implementada \\
    \textit{Csrrsi} & Implementada \\
    \textit{Csrrci} & Implementada \\

    \multicolumn{2}{c}{\textbf{Extensión Zifencei — Sincronización de instrucións}} \\
    \textit{Fence.i} & Implementada \\
\end{longtable}


\section{Implementacións dos pipelines}\label{sec:implt_pipelines}
Neste proxecto non se implementaron as unidades funcionais que están segmentadas como tal, polo que á hora de simular o retardo das instrucións, decidiuse empregar arrays para imitar o proceso dunha instrución atravesando o pipeline. Os ciclos necesarios para saír do array son a latencia, e unha vez fóra, as instrucións son procesadas. Adicionalmente, se nun ciclo a instrución que saiu é un \acrfull{nop}, búscase a anterior para que sexa executada. 

\section{Funcionalidades do simulador}\label{sec:func_sim}
Antes de comezar este proxecto, xa existía unha estrutura base deste simulador, implementando todas as funcionalidades básicas recollidas na base RV32I. Isto inclúe todas as instrucións de lectura e escritura de datos en memoria e rexistros, suma e resta (incluso con operandos inmediatos), operacións lóxicas, saltos e ramas. Esta primeira versión podía executar programas relativamente sinxelos.

Ao finalizar este traballo, engadíronse o módulo de multiplicación para a extensión M, o módulo de operación de punto flotante simple para a extensión F e algunhas instrucións adicionais para as extensións Zicsr e Zifencei. Agora, permite a execución de multiplicacións, divisións e operacións con datos de tipo float, xunto fence e instrucións de tipo \acrshort{csr}.

\section{Ferramentas empregadas}\label{sec:ferramentas}
Durante o proxecto empregáronse 5 ferramentas: Segger, Visual Studio 2022, Git, GTK Wave e SystemC. A continuación, unha breve explicación do seu funcionamento, alternativas dispoñibles e comparativas explicando o porqué desta elección.

\subsection{Segger Embedded Studio for RISC-V}\label{sec:segger}
Segger Embedded Studio for RISC-V é un \acrfull{ide} que permite compilar para RISC-V, incluindo obxectivos concretos como RV32, producir arquivos .elf e ver o código ensamblador. Foi principalmente empregado á hora de escribir código en C para \gls{tests} ou \gls{benchmarks}. Ademais, o depurador permite ver código ensamblador coas direccións, polo que foi realmente útil á hora de encontrar bugs. Existen alternativas populares, como CLion de JetBrains co Toolchain de RISC-V, Visual Studio Code ou Eclipse. No caso de CLion é de pago, polo que é un gran punto en contra. Se ben a universidade ofrece claves, sería necesario engadir o toolchain de RISC-V para poder compilar código para RISC-V, facendo o proceso máis complexo. Visual Studio Code tampouco inclúe ferramentas de base, polo que sería necesario buscar plugins e configurar todo para que sexa apto. Por último, Eclipse cunha configuración avanzada tamén podería ser unha alternativa. Se ben o proceso de instalación non é complexo, non inclúe obxectivos determinados. Todo isto fai que Segger sexa a mellor alternativa, xa que inclúe configuracións xa feitas e todas as ferramentas necesarias sen apenas configuración.

\subsection{Visual Studio 2022}\label{sec:visual_studio}
Á hora de traballar no simulador con C++, o \acrshort{ide} elixido foi Visual Studio 2022. Entre as características máis destacables están: integración con Git, depuración con opcións avanzadas, bo funcionamento con GTK Wave e SystemC, \dots  Existen infinidade de alternativas, pero este foi o elixido por ser a elección máis habitual para este tipo de proxectos polo estudante. Ademais, xa fora empregado na asinatura de Codeseño \gls{hardware}/\gls{software} xunto a SystemC. 

\subsection{GTK Wave}\label{sec:gtkwave}
Para solventar algúns dos problemas máis complexos, foi necesario empregar esta ferramenta. Este software permite, unha vez engadidas trazas no código, rexistrar os cambios de valor de sinais e variables para despois mostralas nun gráfico de ondas. Se ben non é moi popular, xa foi empregada nalgunha asinatura, polo que  coñecela previamente foi imprescindible para elixila.

\subsection{Git}\label{sec:git}
Unha das ferramentas máis empregadas en todos os proxectos é Git. É un sistema de control de versións, polo que mediante repositorios crea un ficheiro onde se almacenan todos os cambios en distintos arquivos. Isto axuda a volver a versións anteriores en caso de erros nas modificacións máis recentes ou evitar a pérdida do traballo en caso de fallo do equipo de traballo.


\subsection{SystemC}\label{sec:imp_systemC}
Como se comentou na sección \ref{sec:systemc}, esta é unha meta-linguaxe creada en C++ e que se pode empregar en calquera entorno de desenvolvemento.



