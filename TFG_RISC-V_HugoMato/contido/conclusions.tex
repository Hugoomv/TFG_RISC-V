\chapter{Conclusións}
\label{chap:conclusions}
 
\lettrine{N}{este} traballo implementáronse novas extensións para a arquitectura RV32I de RISC-V. Agora mesmo, o simulador ten implementadas todas as instrucións das extensións M (multiplicación/división), Zicsr, Zifencei e unha gran parte da extensión F (punto flotante simple). O simulador é útil como ferramenta para guiar implementacións físicas da arquitectura. Durante a realización deste traballo validouse o correcto funcionamento do simulador para todas as instrucións implementadas, e fixéronse probas de rendemento para distintas configuracións do multiplicador.

Como traballo futuro sería interesante engadir máis extensións, particularmente A (instrucións atómicas) e D (punto flotante de dobre precisión). Da mesma forma, sería recomendable modelar unha xerarquía de memoria que inclúa memoria cache, xa que a implementación actual da memoria segue unha arquitectura Harvard con unha latencia de 1 ciclo. Tamén se podería incluír soporte para 64-bits e formato comprimido de instrucións. 