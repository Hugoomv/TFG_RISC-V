\chapter{Conclusións}
\label{chap:conclusions}

\lettrine{D}{erradeiro} capítulo da memoria, onde se presentará a situación final do traballo, as leccións aprendidas, a relación coas competencias da titulación en xeral e a mención en particular, posibles liñas futuras,\dots

\section{Traballo futuro}\label{sec:traballo_futuro}
Agora mesmo, o simulador ten implementadas todas as instrucións das extensións M (multiplicación/división), Zicsr, Zifencei e unha gran parte da extensión F (punto flotante simple). Poderíanse engadir máis extensións, como a A, D ou L. Da mesma forma, sería interesante simular a memoria e a súa xerarquía. A implementación actual non é moi realista xa que se trata dun sinxelo módulo que sempre escribe ou lee nun só ciclo. Tamén sería interesante incluir soporte para 64-bits, coa posibilidade de seguir parametrizando o simulador, polo que, por exemplo, o tamaño dos rexistros ou o funcionamento do módulo de decodificación veríanse afectados segundo a base empregada.


\section{Presuposto}\label{sec:presuposto}
Para determinar o custo do desenvolvemento do proxecto sería necesario calcular primeiro as horas dedicadas por parte do estudante e do titor e coñecer os prezos destas. Para iso, o BOE ~\cite{boe} establece uns valores mínimos que resultan en 15€/hora para o estudante e 19€/hora para o titor. Na táboa \ref{tab:salario} vése o resultado dos cálculos pertinentes.

\begin{table}[hp!]
    \centering
    \rowcolors{2}{white}{udcgray!25}
    \begin{tabular}{c|c|c|c}
    \rowcolor{udcpink!25}
    \textbf{Persoa} & \textbf{horas}  & \textbf{custo/h} & \textbf{total} 
    \\\hline
    \textit{Estudante} & 240 & 15€/h & 3.600€\\
    \textit{Titor} & 35 & 19€/h & 665€\\
    \multicolumn{3}{c|}{\textbf{Resultado}} & \textit{4.265€} \\
    \end{tabular}
    \caption{Táboa co custo/h e total dos traballadores implicados.}
    \label{tab:salario}
\end{table}

Ademais, é necesario engadir os gastos en licenzas e materiais inventariables. Neste caso, todas as ferramentas empregadas foron gratuítas e non se empregaron materiais adicionais máis aló do portátil, cun valor de 600€. O resultado final móstrase na táboa \ref{tab:precio_final}, sendo de 4.865€.

\begin{table}[hp!]
    \centering
    \rowcolors{2}{white}{udcgray!25}
    \begin{tabular}{c|c}
    \rowcolor{udcpink!25}
    \textbf{Recurso} & \textbf{Prezo} 
    \\\hline
    \textit{Man de obra} & 4.265€\\
    \textit{Materiais e licenzas} & 600€\\
    \textbf{Resultado} & 4.865€ \\
    \end{tabular}
    \caption{Táboa co resultado do custo total do proxecto.}
    \label{tab:precio_final}
\end{table}