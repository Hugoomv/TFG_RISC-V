\chapter{Conclusións}
\label{chap:conclusions}

\lettrine{D}{erradeiro} capítulo da memoria, onde se presentará a
situación final do traballo, as leccións aprendidas, a relación coas competencias da titulación en xeral e a mención en particular,
posibles liñas futuras,\dots

\section{Resultados}\label{chap:resultados}
Tras finalizar o proxecto, pódese garantir que engadir novas extensións coas súas correspondentes novas instrucións non comprometen o traballo anterior. O funcionamento do resto de módulos segue sendo correcto, o rendemento non se viu deteriorado en ningún momento.

Por outra parte, a posibilidade de modificar as latencias dalgunhas instrucións grazas á parametrización engadida, mostra como cambia o rendemento no conxunto dun programa. Por exemplo, á hora de executar o benchmark SPMV para que realice multiplicación de enteiros, obtemos resultados moi interesantes segundo as latencias. Como vemos na táboa \ref{tab:rendemento_spmv}, 

\begin{table}[hp!]
    \centering
    \rowcolors{2}{white}{udcgray!25}
    \begin{tabular}{c|c|c}
    \rowcolor{udcpink!25}
    \textbf{Modificacións realizadas} & \textbf{Número de instrucións}  & \textbf{Tempo} 
    \\\hline
    \textit{SPMV base} & a & a \\
    \textit{Latencia de mul = 5} & a & b\\
    \textit{Latencia de mul = 10} & a & b\\
    \textit{Latencia de mulh = 5} & a & b\\
    \textit{Latencia de mul = 5 e mulh = 5} & a & b\\
    \end{tabular}
    \caption{Rendemento do benchmarks SPMV segundo as latencia de distintas operacións.}
    \label{tab:rendemento_spmv}
\end{table}

Finalmente, destacar que a velocidade de simulación deste proxecto en comparación coa que se podería obter se VHDL ou Verilog fose empregado é moi superior.

\section{Traballo futuro}\label{chap:traballo_futuro}
Agora mesmo, o simulador inclúe todas as instrucións implementadas ata o que é conecido como extensión 'G'. Poderíanse engadir máis extensións, así como unha memoria ¿? ou incluir soporte para 64bits.


