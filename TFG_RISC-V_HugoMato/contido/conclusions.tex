\chapter{Conclusións}
\label{chap:conclusions}

\lettrine{D}{erradeiro} capítulo da memoria, onde se presentará a situación final do traballo, as leccións aprendidas, a relación coas competencias da titulación en xeral e a mención en particular, posibles liñas futuras,\dots

\section{Traballo futuro}\label{sec:traballo_futuro}
Agora mesmo, o simulador ten implementadas todas as instrucións das extensións M (multiplicación/división), Zicsr, Zifencei e unha gran parte da extensión F (punto flotante simple). Poderíanse engadir máis extensións, como a A, D ou L. Da mesma forma, sería interesante simular a memoria e a súa xerarquía. A implementación actual non é moi realista xa que se trata dun sinxelo módulo que sempre escribe ou lee nun só ciclo. Tamén sería interesante incluir soporte para 64-bits, coa posibilidade de seguir parametrizando o simulador, polo que, por exemplo, o tamaño dos rexistros ou o funcionamento do módulo de decodificación veríanse afectados segundo a base empregada.
