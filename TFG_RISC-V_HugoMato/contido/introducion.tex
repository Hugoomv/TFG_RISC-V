\chapter{Introdución}
\label{chap:introducion}

\lettrine{E}{ste} proxecto busca crear un simulador de RISC-V empregando a librería SystemC en C++. Ao longo desta memoria describiranse as etapas de execución, os distintos módulos e a motivación destes así como as extensións.


%%citas~\cite{ErlangBook,ErlangWebBook} e de referencias
%%internas (sección \ref{sec:mostra}, páxina \pageref{sec:mostra}).


\section{Motivación}
\label{sec:motivación}
RISC-V apunta a ser unha das arquitecturas máis empregadas nun futuro, xa que é libre, permitindo aforrar o custo de licenzas. Grazas a que se pode modificar, engadindo ou eliminando funcionalidades, isto permite que abarque múltiples sectores, dende chips máis sinxelos orientados a IoT, ata competir con ARM en sistemas embebidos. REFES AQUI\\
Durante o proceso de deseño, unha parte clave é a verificación do correcto funcionamento. REFES AQUI Se ben é posible crear un chip con cada versión, na práctica, debido aos longos tempos e altos custos, é inviable. Ahí é onde un simulador toma protagonismo, xa que permite probar de forma rápida, simple e barata os deseños creados. 

\section{Obxectivos}
\label{sec:obxectivos}
Os obxectivos deste proxecto son modelar e simular, usando SystemC, as seguintes extensións da arquitectura RV32I: 

\begin{itemize}
    \item multiplicación e división por números enteiros (extensión M).
    \item aritmética en punto flotante de simple (extensión F).
    \item  operacións atómicas  (extensión A).
    \item  xestión de rexistros de control e estado (extensión Zicsr).
    \item sincronización de escritura de instrucións (extensión Zifencei).
\end{itemize}

O modelado será totalmente parametrizable, permitindo especificar a latencia das diferentes instrucións. Tamén permitirá especificar o número de canles de execución para as unidades de enteiros a punto flotante, e se estes están ou non totalmente segmentados. 

Desta maneira, a simulación permitirá comparar o rendemento de, por exemplo, unha implementación na que multiplicador e divisor comparten circuítos, cunha na que ambos son independentes, e tamén comparar un divisor totalmente segmentado con un que non o sexa. 

Os resultados do modelado e a simulación son dous: verificar o correcto funcionamento da arquitectura, e comprobar o seu rendemento. 


\section{Metodoloxía}
\label{sec:metodoloxía}
O método de traballo será incremental, dividindo as tarefas en partes independentes que van ser implementadas, simuladas e verificadas por orde de complexidade antes de proceder coa seguinte. \\
O procedemento habitual é unha reunión semanal na que se revisa o feito na semana anterior, acompañado de probas cos tests correspondentes para esa parte. Despois, decídese cal é o seguinte paso, podendo ser a implementación dunha nova extensión ou modificar un módulo do simulador.

\subsection{Fases principais}
\begin{itemize}
    \item Estudio da documentación existente sobre RISC-V.
    \item Familiarización coa implementación base de RV32I en SystemC. 
    \item Modelado e simulación do multiplicador e divisor de enteiros. 
    \item Modelado e simulación das extensións de punto flotante F e D. 
    \item Modelado e simulación de extensións Zicsr e  Zifencei.
    \item Empaquetamento do software. 

\end{itemize}

\section{Contida da memoria}
\label{sec:contido_memoria}
Texto aquí :)

