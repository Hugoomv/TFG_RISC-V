%%%%%%%%%%%%%%%%%%%%%%%%%%%%%%%%%%%%%%%%%%%%%%%%%%%%%%%%%%%%%%%%%%%%%%%%%%%%%%%%

\pagestyle{empty}
\begin{abstract}
  RISC-V é unha nova arquitectura libre para procesadores programables. Está chamada a competir con outras arquitecturas máis establecidas como ARM en moitos ámbitos tecnolóxicos. Ademais de que as especificacións de RISC-V son abertas, a principal vantaxe desta arquitectura é que cubre desde as implementacións máis sinxelas para sistemas embarcados, ata as máis potentes para cálculo científico e multimedia. Todo isto é posible mediante a especificación de moitas das características máis avanzadas como extensións ás arquitecturas básicas. 

A implementación dun procesador require previamente dun modelado e simulación que garantan que o funcionamento final vai ser o correcto. 

Neste traballo, pártese dun modelo da versión básica RV32I realizada en SystemC. A única extensión implementada é a multiplicación para enteiros. O obxectivo desde proxecto é modelar e simular extensións adicionais ata chegar ao coñecido como nivel G. 

  \vspace*{25pt}
  \begin{segundoresumo}
    \blindtext % substitúe este comando polo resumo do teu TFG
               % na lingua secundaria do documento (tipicamente: inglés)
  \end{segundoresumo}
\vspace*{25pt}
\begin{multicols}{2}
\begin{description}
\item [\palabraschaveprincipal:] \mbox{} \\[-20pt]
  \begin{itemize}
      \item RISC-V
      \item Simulador
      \item SystemC
      \item Procesador
      \item Arquitectura
      \item Extensións
      \item C++
  \end{itemize}
\end{description}
\begin{description}
\item [\palabraschavesecundaria:] \mbox{} \\[-20pt]
  \blindlist{itemize}[7] % substitúe este comando por un itemize
                         % que relacione as palabras chave
                         % que mellor identifiquen o teu TFG
                         % no idioma secundario da memoria (tipicamente: inglés)
\end{description}
\end{multicols}

\end{abstract}
\pagestyle{fancy}

%%%%%%%%%%%%%%%%%%%%%%%%%%%%%%%%%%%%%%%%%%%%%%%%%%%%%%%%%%%%%%%%%%%%%%%%%%%%%%%%
