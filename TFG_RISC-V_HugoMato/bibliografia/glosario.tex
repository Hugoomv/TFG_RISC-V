%%%%%%%%%%%%%%%%%%%%%%%%%%%%%%%%%%%%%%%%%%%%%%%%%%%%%%%%%%%%%%%%%%%%%%%%%%%%%%%%
% Obxectivo: Lista de termos empregados no documento,                          %
%            xunto cos seus respectivos significados.                          %
%%%%%%%%%%%%%%%%%%%%%%%%%%%%%%%%%%%%%%%%%%%%%%%%%%%%%%%%%%%%%%%%%%%%%%%%%%%%%%%%

\newglossaryentry{bytecode}{
  name=bytecode,
  description={Código independente da máquina que xeran compiladores de determinadas linguaxes (Java, Erlang,\dots) e que é executado polo correspondente intérprete.}
}

\newglossaryentry{hazards}{
    name=hazard,
    description={Risco producido por unha dependencia RAW,WAR ou WAW. Pode chegar a causar erros na execución dun programa informático.}
}

\newglossaryentry{Software}{
    name=software,
    description={Conxunto de compoñentes lóxicos que permiten realizar determinadas funcións nun equipo tecnolóxico.}
}

\newglossaryentry{Hardware}{
    name=hardware,
    description={}
}

\newglossaryentry{benchmarks}{
    name=benchmark,
    description={Examen que se realiza coa fin de comprobar que un programa funciona sen erros e producindo a saída correcta.}
}

\newglossaryentry{tests}{
    name=test,
    description={Proba mediante a cal se revisa que o funcionamento dun programa é o esperado. Tipícamente, busca simular casos reais de execución.}
}

\newglossaryentry{arquitecturas}{
    name=arquitectura,
    description={No contexto da informática, refírese ao conxunto de circuítos e outros compoñentes dos que se compón un sistema.--REV}
}

\newglossaryentry{depuración}{
    name=depuración,
    description={Proceso durante o que se revisa o funcionamento dun programa informático, comprobando que o funcionamento é o esperado e correcto.}
}

\newglossaryentry{bits}{
    name=bits,
    description={Unidade mínima de información en informática, é un díxito do sistema binario, polo que pode valer 0 ou 1.  }
}

\newglossaryentry{rexistro}{
    name=rexistro,
    description={ }
}

\newglossaryentry{chips}{
    name=chip,
    description={ }
}