%%%%%%%%%%%%%%%%%%%%%%%%%%%%%%%%%%%%%%%%%%%%%%%%%%%%%%%%%%%%%%%%%%%%%%%%%%%%%%%%
% Obxectivo: Lista de termos empregados no documento,                          %
%            xunto cos seus respectivos significados.                          %
%%%%%%%%%%%%%%%%%%%%%%%%%%%%%%%%%%%%%%%%%%%%%%%%%%%%%%%%%%%%%%%%%%%%%%%%%%%%%%%%

\newglossaryentry{hazards}{
    name=hazard,
    description={Risco producido por unha dependencia RAW, WAR ou WAW. Pode chegar a causar erros na execución dun programa informático}
}

\newglossaryentry{software}{
    name=software,
    description={Conxunto de compoñentes lóxicos que permiten realizar determinadas funcións nun equipo tecnolóxico}
}

\newglossaryentry{hardware}{
    name=hardware,
    description={Partes físicas dun sistema informático, formado por todos os compoñentes electrónicos, circuítos e periféricos}
}

\newglossaryentry{benchmarks}{
    name=benchmarks,
    description={Examen que se realiza coa fin de comprobar que un programa funciona sen erros e producindo a saída correcta}
}

\newglossaryentry{tests}{
    name=tests,
    description={Proba mediante a cal se revisa que o funcionamento dun programa é o esperado. Tipícamente, busca simular casos reais de execución}
}

\newglossaryentry{arquitectura}{
    name=arquitecturas,
    description={No contexto da informática, refírese ao deseño e estructura dun conxunto de circuítos e outros compoñentes dos que se compón un sistema}
}

\newglossaryentry{depuración}{
    name=depuración,
    description={Proceso durante o que se revisa o funcionamento dun programa informático, comprobando que o funcionamento é o esperado e correcto}
}

\newglossaryentry{bits}{
    name=bits,
    description={Unidade mínima de información en informática, é un díxito do sistema binario, polo que pode valer 0 ou 1}
}

\newglossaryentry{rexistros}{
    name=rexistros,
    description={Pequenos almacéns de información temporal, caracterizados por estar dentro da CPU e a súa alta velocidade de traballo}
}

\newglossaryentry{chips}{
    name=chips,
    description={ Conxunto de circuítos integrados dentro dunha pequena peza de material semiconductor, co que se realizan varias funcións en ordenadores e outros dispositivos electrónicos}
}

\newglossaryentry{meta-linguaxe}{
    name=meta-linguaxe,
    description={ Engadir funcionalidades á unha linguaxe mediante librarías e conxuntos de macros}
}